%! Author = Matthew
%! Date = 16.05.2022
\documentclass[12pt,aspectratio=169,svgnames]{beamer}

\usepackage{modules/cgs}

\usetikzlibrary{calc}

\begin{document} \maketitle

\begin{frame}{Содержание}
	\tableofcontents
\end{frame}

\section{Задачи, выполнимые за $O(1)$}
\begin{frame}{Некоторые договоренности}

	Для удобства мы будем считать, что точки , с которыми мы работаем на плоскости или в пространстве
	\alert{общего положения}, то есть, что для них выполняется следующее

	\begin{itemize}
		\item Никакие 3 из них не лежат на одной прямой.

		\item Никакие 4 из них не лежат на одной окружности.

		\item У них нет общих $x$ и $y$ координат.
	\end{itemize}

	В совокупности это верно почти всегда.

\end{frame}

\begin{frame}{Задание фигур уравнениями}

	При работе с геометрическими объектами удобно задавать их уравнениями

		\begin{itemize}
			\item Прямая:\\

			$\ell\colon ax + by + c = 0$ или $\ell\colon y = kx + b$.\\
			Прямая через точки $A(x_1, y_1), B(x_2, y_2)$:
			\[ \ell\colon \begin{cases} ax_1 + by_1 + c = 0 \\ ax_2 + by_2 + c = 0 \end{cases}\]
			\[ a = y_2 - y_1, \ b = x_1 - x_2, \ c = -(ax_1 + by_1)\]

			\item Окружность:
			\[ (x - x_0)^2 + (y - y_0)^2 = R^2, \text{ где } (x_0, y_0) \text{ -- центр, а } R \text{ -- радиус} \]
		\end{itemize}

\end{frame}

\begin{frame}{Угол между векторами}

	Рассмотрим вектора $\vec{a} = (x_1, y_1)$ и $\vec{b} = (x_2, y_2)$.
	\[ \langle \vec{a}, \vec{b} \rangle = x_1 x_2 + y_1 y_2 = |\vec{a}||\vec{b}| \cos{\angle{(\vec{a}, \vec{b})}} = \sqrt{x_1^2 + y_1^2}\sqrt{x_2^2 + y_2^2} \cos{\angle(\vec{a}, \vec{b})}\]

	Значит, мы знаем, как найти угол между векторами $\vec{a}$ и $\vec{b}$:
	\[ \angle(\vec{a}, \vec{b}) = \arccos\bigg(\frac{x_1 y_1 + x_2 y_2}{\sqrt{x_1^2 + y_1^2} \sqrt{x_2^2 + y_2^2}}\bigg)\]

\end{frame}

\begin{frame}{Косое произведение}
	\begin{defn}
		\alert{Косым проивезеднием} векторов $\vec{a} = (x_1, y_1)$ и $\vec{b} = (x_2, y_2)$ на плоскости будем называть
		\[ \vec{a} \wedge \vec{b} =  \begin{vmatrix} x_1 & y_2 \\ x_2 & y_2  \end{vmatrix} = x_1 y_2 - x_2 y_1\]
	\end{defn}

	Покажем, что $\vec{a} \wedge \vec{b} = |\vec{a}| |\vec{b}| \sin{\angle(\vec{a}, \vec{b})}$, где $\angle(\vec{a}, \vec{b})$ -- угол
	вращения против часовой стрелки от $\vec{a}$ к $\vec{b}$.
\end{frame}

\begin{frame}{Эквивалентность определений}

	В самом деле, $\cos(\angle(\vec{a}, \vec{b})) = |x_1 x_2 + y_1 y_2| / (\sqrt{x_1^2 + y_1^2}\sqrt{x_2^2 + y_2^2})$.
	\begin{multline*} \sin(\angle(\vec{a},\vec{b})) = \sqrt{1 - \cos^2({\angle(\vec{a}, \vec{b})}}) = \sqrt{1 - \frac{(x_1 x_2 + y_1 y_2)^2}{(x_1^2 + y_1^2)(x_2^2 + y_2^2)}} =\\= \sqrt{\frac{x_1^2 x_2^2 + x_1^2 y_2^2 + y_1^2 x_2^2 + y_1^2 y_2^2 - x_1^2 x_2^2 - y_1^2 y_2^2 - 2 x_1 x_2 y_1 y_2 }{(x_1^2 + y_1^2)(x_2^2 + y_2^2)}} \\= \frac{x_1 y_2 - y_1 x_2}{\sqrt{(x_1^2 + y_1^2)}\sqrt{(x_2^2 + y_2^2)}} = \frac{\vec{a} \wedge \vec{b}}{|\vec{a}||\vec{b}|} \end{multline*}

\end{frame}

\begin{frame}{Площадь треугольника}

	Теперь ясно, что $\vec{a} \wedge \vec{b} = \frac{1}{2} S_{\triangle{ABC}}$, где площадь \alert{ориентированная}.

	\begin{center}
	\begin{tikzpicture}
	\coordinate[label=left:$A$] (A)at(0,0); \coordinate[label=left:$B$](B)at(1,3); \coordinate[label=right:$C$](C)at(4,3); \coordinate[label=right:$D$](D)at(3,0);
	\draw[->, fill4,ultra thick] (0,0)--(1,3) node[midway, left = 0.1]{$\vec{a}$};
	\draw[->, fill4,ultra thick] (0,0)--(3,0) node[midway, below = 0.1]{$\vec{b}$};
	\draw[fill4, thick] (1,3)--(3,0);
	\draw[fill4, fill=fill4, fill opacity=0.5, draw opacity = 0] plot coordinates{(0,0) (1,3) (3, 0)}--cycle;
	\draw (A)--(B)--(C)--(D)--cycle;

	\end{tikzpicture}
	\end{center}

\end{frame}
\begin{frame}
	\textsc{Скоро допишу, просто попробовал подгрузить.}
\end{frame}

\end{document}
