\documentclass[12pt,aspectratio=169,svgnames]{beamer}

\usepackage{modules/cgs-e}

\begin{document} \maketitle

\begin{frame}{Table of contents}
	\tableofcontents
\end{frame}

\section{Theses and basics}

\begin{frame}{Ontological thesis}
	A view that the things that are discussed do not exist physically, \\
	yet obey certain rules, and we can therefore describe them \\
	rightfully and completely.
\end{frame}

\begin{frame}{Linguistic thesis}
	An opposing view: the statements that are discussed do not have \\
	the purpose to reflect factial trutth, but are rather \\
	useful fictions.
\end{frame}

\begin{frame}{A basic description of fictionalism}
	All the claims that are made are technically false, \\
	still they are worth considering, since these exact claims \\
	might be needed for some theoretical purposes. \bigskip

	Simply said, fictionalism aligns with the linguistic thesis.
\end{frame}

\section{Fictionalism in mathematics: departure points}

\begin{frame}{Fictionalism in mathematics}
	There are no such objects physically as numbers, rings, \\
	fields, manifolds, etc. Therefore any statements \\
	about them are untrue, because the objects are made up. \\
	It does not mean though that mathematics is pure fiction.
\end{frame}

\begin{frame}{Countributors to Fictionalism}
\begin{itemize}
	\item Field (1980, \ldots, 2016)
	\item Balaguer (1996, \ldots, 2009)
	\item Leng (2005, 2010)
	\item Rosen (2001)
	\item Hoffman (2004)
\end{itemize} \bigskip

	Quite recent isn't it?..
\end{frame}

\section{An inference of fictionalism}

\begin{frame}{Face value} \label{sl:fv}
	Mathematical sentences should be read at face value, \\
	they are making straightforward claims \\
	about the nature of certain objects.
\end{frame}

\begin{frame}{Existence of underlying objects} \label{sl:uex}
	If the sentence “4 is even” is true, then the object \\
	it is about must exist. \bigskip

	\textcolor{black!35!white}{
		Possible fix: if 4 exists, it evenness does not contradict \\
		any axioms we have introduced.}
\end{frame}

\begin{frame}{Existence of mathematical objects} \label{sl:mex}
	If sentences like “4 is even” are true, then \\
	there exist mathematical objects.
\end{frame}

\begin{frame}{Existence of abstract objects} \label{sl:abstract}
	If there are mathematical objects, then there exist \\
	{\it abstract} objects~— ones that do not occupy \\
	a point in space or time.
\end{frame}

\begin{frame}{Abstract objects do not exist} \label{sl:noabstract}
\begin{itemize}
	\item There are no such things as abstract objects,
	\item Therefore, there are no such things as mathematical objects,
	\item Therefore, sentences like “4 is even” are not true.
\end{itemize}
\end{frame}

\section{Alternatives to fictionalism}

\begin{frame}{Why the aforementioned inference is important}
	The logical tools used for deriving the statements are \\
	indeed valid. Therefore, to disagree with a fictionalist, \\
	one has to disagree with one of the premises we used.
\end{frame}

\begin{frame}{Alternatives to fictionalism}
\begin{itemize}
	\item Mathematical statements are not straightforward \\
	claims~(\ref{sl:fv})~— {\bf paraphrase nominalists;} \bigskip

	\item Truth of a statement does not imply \\
	the existence of the underlying object~(\ref{sl:uex})~— \\
	{\bf deflationary-truth nominalists;}
\end{itemize}
\end{frame}

\begin{frame}{Alternatives to fictionalism}
\begin{itemize}
	\item The existence of mathematical objects \\
	does not imply the existence of abstract objects~(\ref{sl:abstract})~—\\
	{\bf physicalists} or {\bf psychologists;} \bigskip

	\item There exist abstract objects~(\ref{sl:noabstract})~—\\
	{\bf platonists.}
\end{itemize}
\end{frame}

\section{Objections to fictionalism}

\begin{frame}{Indispensability argument}
	Mathematical statements help other theories like physics, \\
	and the description of the world given by physics is pretty \\
	accurate, therefore we have quite a good reason \\
	to distinguish true mathematical statements.
\end{frame}

\begin{frame}{Objectivity}
	Among mathematical statements there are some of \\
	different flavors, like “4 is even” seems true, and “3 is composite” \\
	seems false. Seems~— aligns with the theory. However, fictionalists \\
	are obliged to say that both statements are false. \bigskip

	\textcolor{black!35!white}{
		There are fictionalist responses to this of course.}
\end{frame}

\begin{frame}{Similarity to fiction}
	Consistency is an important criterion for mathematical statements, \\
	which is not the case for a set of any made-up ones. \bigskip

	\textcolor{black!35!white}{
		Fictionalism does not involve the claim that there are no \\
		important disanalogies between mathematics and fiction.}
\end{frame}

\begin{frame}{Accepting and believing}
	Fictionalists acknowledge the correctness of mathematical \\
	statements, but do not approve of their truth. Does it mean \\
	that they accept them without believing them? \bigskip

	Why not then say that truth\(=\)acceptance by fictionalists?
\end{frame}

\section{Conclusion}

\begin{frame}{Conclusion}
	Fictionalism is a concept based on that mathematics needs \\
	the existence of any object it is talking about and imply \\
	it directly~— therefore its statements can not be considered true. \bigskip

	Mathematics, though, usually say something like “would \\
	an object exist that obeys certain axioms, \\
	the following would hold for it”. \bigskip

	{\bf Thanks!}
\end{frame}

\end{document}
