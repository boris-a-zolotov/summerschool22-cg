\section{Проблема локализации точки}

\begin{frame}{Постановка задачи}
	Дано разбиение плоскости на области, построить структуру данных, \\
	которая по точке будет указывать на её область.
	\vspace{-1mm}

  \begin{center} \begin{tikzpicture}[yscale=0.47,xscale=0.54]
	\PutCoordinates
	\DrawDcel
	\foreach \i in {1,...,15} {\placecrd{\i}}
  \end{tikzpicture} \end{center}
\end{frame}


\begin{frame}{Vertical slabs}
	Проведём через каждую вершину вертикальную прямую, \\
	подразбив области. Внутри полоски набор областей фиксирован.
	\vspace{-1mm}

  \begin{center} \begin{tikzpicture}[yscale=0.47,xscale=0.54]
	\PutCoordinates
	\DrawDcel
	\foreach \i in {1,...,15} {\placecrd{\i}}
	\foreach \i in {1,...,15} {\vertslab{\i}}
  \end{tikzpicture} \end{center}
\end{frame}


\begin{frame}{Разбиение внутри полосок}
	Как-нибудь {\it (sweeping line)} сохраним список отрезков, \\
	подразбивающих каждую область по вертикали.
	\vspace{-1mm}

  \begin{center} \begin{tikzpicture}[yscale=0.47,xscale=0.54]
	\PutCoordinates
	\DrawDcel
	\DrawDividing
	\foreach \i in {1,...,15} {\placecrd{\i}}
	\foreach \i in {1,...,15} {\vertslab{\i}}
  \end{tikzpicture} \end{center}
\end{frame}


\begin{frame}{Двоичный поиск}
	Сначала по координате находим вертикальную полоску, \\
	потом, выясняя, выше или ниже точка отрезка,~— область.
	\vspace{-1mm}

  \begin{center} \begin{tikzpicture}[yscale=0.47,xscale=0.54]
	\PutCoordinates
	\DrawDcel
	\DrawDividing
	\foreach \i in {1,...,15} {\placecrd{\i}}
	\foreach \i in {1,...,15} {\vertslab{\i}}
	\DrawTree
  \end{tikzpicture} \end{center}
\end{frame}


\begin{frame}{Анализ}
	Запрос~— \(O\lr*{\log n}\). На построение требуется \(O\lr*{n^2}\) памяти \\
	и времени. Как улучшить~— {\it persistent trees.}
	\vspace{-1mm}

  \begin{center} \begin{tikzpicture}[yscale=0.47,xscale=0.54]
	\PutCoordinates
	\DrawDcel
	\DrawDividing
	\foreach \i in {1,...,15} {\placecrd{\i}}
	\foreach \i in {1,...,15} {\vertslab{\i}}
	\DrawTree
  \end{tikzpicture} \end{center}
\end{frame}
