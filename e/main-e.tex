\documentclass[12pt,aspectratio=169,svgnames]{beamer}

\usepackage{modules/cgs-e}

\begin{document} \maketitle

\begin{frame}{Table of contents}
	\tableofcontents
\end{frame}

\section{Myself}

\begin{frame}{About Myself}
\begin{itemize}
	\item I got my master's degree from SPbU in 2021,
	\item I have scientific ties with ULB (Brussels), BMS (Berlin),
	\item I also possess decent teaching experience.
\end{itemize} \vspace{4mm}

	Today I am going to be talking about \\ the research areas that I pursue.

\end{frame}

\section{Voronoi Diagrams: Known Results}

\begin{frame}{Chan's structure}
	One might not want to know where the vertices and \\
	the edges of a Voronoi diagram are. Instead, they might be \\
	interested to, given a query point \(q\), specify \\
	the cell \(f\) such that \(q \in f\). \bigskip

	This is known as the {\it Implicit Voronoi Diagram} problem. \\
	There is a data structure (called {\it Chan's structure)} \\
	devised to solve it and several related problems really fast.
\end{frame}

\begin{frame}{Number of changes}
	It has been proved that when a new site is added \\
	to a Voronoi diagram, there is \(O\lr*{\sqrt{n}}\) combinatorial \\
	changes (additions and deletions of edges) \\
	that the diagram undergoes. \bigskip

	What is more, this remains true for a diagram \\
	one of whose sites is a line.
\end{frame}

\begin{frame}{Sublinear algorithm}
	An algorithm is known that implements the combinatorial \\
	changes a Voronoi diagram undergoes \\
	when a site is inserted. \bigskip

	The algorithm works in time \(O\lr*{n^{3/4}}\). There still remains \\
	a gap between \(O\lr*{n^{3/4}}\) and \(O\lr*{n^{1/2}}\).
\end{frame}

\section{Voronoi Diagrams: Pending Results}

\begin{frame}{Planned to be done}
\begin{itemize}
	\item Close the gap between \(O\lr*{n^{3/4}}\) and \(O\lr*{n^{1/2}}\): \\
		find an optimal-time algorithm or prove that the optimal \\
		time bound is actually greater. \bigskip
	\item Generalize the techniques that are applied \\
		to Voronoi diagrams over other metrics.
\end{itemize}
\end{frame}

\section{Nets: Known Results}

\begin{frame}{Gluing tree}
	\vphantom{x}{\it The gluing tree} is a way to describe how the boundary \\
	of a polygon is glued to itself to form the surface \\
	of a convex polyhedron. \bigskip

	Using gluing trees allowed for showing that there is \\
	a polygon (a star, namely) with \(2n\) vertices that can be glued \\
	into a convex polyhedron in \(\Omega\lr*{2^n}\) \\
	combinatorially distinct ways.
\end{frame}

\begin{frame}{Other regular polygons}
	For \(n=5\), \(n=5\), \(n>6\) it is known how many (and which \\
	exactly) various polyhedra can be glued if several \\
	congruent regular \(n\)-gons are taken. \bigskip

	Even the gluings that are not edge-to-edge are listed \\
	for \(n>6\) {\it (why is that?)}.
\end{frame}

\section{Nets: Pending Results}

\begin{frame}{Planned to be done}
	Any gluing of triangles or hexagons that produces \\
	a convex polyhedron can be interpreted as a drawing \\
	of a set of polygons whose vertices align \\
	with the triangular grid with \\
	additional properties imposed. \bigskip

	This allows for an easy upper bound on the number of gluings. \\
	To find a lower bound, though, one has to assemble \\
	a series of polyhedra that can be glued.
\end{frame}

\section{PCB}

\begin{frame}{PCB}
	Inspired by industrial projects, the problem of connecting \\
	\(n\) pairs of points in a metric space composed of \\
	several layers, the turns of the paths being restricted \\
	to \(45^\circ\), seems to be NP-hard.
\end{frame}

\begin{frame}{Planned to be done}
\begin{itemize}
	\item Find a polynomial time approximation algorithm \\
		for the PCB routing problem.
	\item Find weights for the parameters such that the corresponding \\
		multicriterion optimisation problem outputs results \\
		that are applicable in practice.
	\item Find algorithms that count in obstacles, static or \\
	posed by the previous paths.
\end{itemize}
\end{frame}

\section{Conclusion}

\begin{frame}{Contribution}
	The corresponding results could be the cornerstones of \\
	the corresponding fields. They make us closer to finding \\
	constructive solutions to the Alexandrov's problem and help \\
	plan routes and process geodesic data faster and better. \bigskip

	The problem of PCB routing is tightly connected with several \\
	projects that need feasible routing algorithms to work with.
\end{frame}

\end{document}
