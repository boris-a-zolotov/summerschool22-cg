\documentclass[12pt,aspectratio=169,svgnames]{beamer}

\usepackage{modules/cgs}

\begin{document} \maketitle

\begin{frame}{Содержание}
	\tableofcontents
\end{frame}

\section{Склейки: важные примеры}

\begin{frame}{Не существует замкнутой формулы}
	Координаты правильного \(N\)-угольника со стороной \(L\) \\
	не выражаются в радикалах. \vspace{-3mm}

	\begin{center} \tikz{
\begin{scope}[scale=0.7]
% \filldraw[draw=white,fill=white] (-1.6,-2.4) rectangle (1.6,3.0);

\begin{scope}[yscale=0.37,xscale=1]
\draw [dotted] (-1.4,0) arc (180:90:1.4cm) arc (90:0:1.4cm);

\coordinate (A) at (0:1.4cm);
\coordinate (B) at (42:1.4cm);
\coordinate (C) at (81:1.4cm);
\coordinate (D) at (128:1.4cm);
\coordinate (E) at (162:1.4cm);
\coordinate (F) at (182:1.4cm);
\coordinate (G) at (224:1.4cm);
\coordinate (H) at (258:1.4cm);
\coordinate (I) at (302:1.4cm);
\coordinate (J) at (325:1.4cm);
\end{scope}

\coordinate (U) at (0,2.44);
\coordinate (V) at (0,-2.44);

\draw[thick,dashed] (A) -- (B) -- (C) -- (D) -- (E);

\draw[thick,dashed] (V) -- (B) -- (U);
\draw[thick,dashed] (V) -- (C) -- (U);
\draw[thick,dashed] (V) -- (D) -- (U);

\draw[thick,dashed] (V) -- (E);

\zal 3 (F) -- (E) -- (U) -- (A) -- (V) -- cycle;

\draw[thick] (E) -- (F) -- (G) -- (H) -- (I) -- (J) -- (A);

\draw[thick] (V) -- (F) -- (U);
\draw[thick] (V) -- (G) -- (U);
\draw[thick] (V) -- (H) -- (U);
\draw[thick] (V) -- (I) -- (U);
\draw[thick] (V) -- (J) -- (U);
\draw[thick] (V) -- (A) -- (U);

\draw[thick] (E) -- (U);

\begin{scope}[yscale=0.37,xscale=1]
\draw [dotted] (-1.4,0) arc (180:270:1.4cm) arc (270:360:1.4cm);
\end{scope}\end{scope}} \end{center} \vspace{-4mm}

	Отсюда координаты вершин бипирамиды (которая получается\\
	из набора треугольников с целыми сторонами по теореме\\
	Александрова) также не выражаются.
\end{frame}

\begin{frame}{Экспоненциально много различных склеек} \ \\ [-0.2cm]
   \begin{thm}
	Для любого чётного \(n\) существует многоугольник\\
	с \(n\) вершинами, из которого можно сделать не менее \(2^{C \cdot n}\)\\
	комбинаторно различных \alert{порёберных} склеек.
   \end{thm}
   \begin{center} \begin{tikzpicture}
	\filldraw[draw=white,fill=PaleGreen,fill opacity=0.6] (1.9,0)
	\foreach \t in {1,...,16} {-- (360/16*\t-11.25 : 0.25) coordinate[midway](mm\t) -- (360/16*\t : 1.9)};
	\draw (mm1) node[circle,draw=white,fill=dgray,inner sep=0.25mm]{ } node[above,inner sep=0.6mm]{{\scriptsize \(v_1\)}};
	\draw (mm9) node[circle,draw=white,fill=dgray,inner sep=0.25mm]{ } node[below,inner sep=0.8mm]{{\scriptsize \(v_2\)}};
   \end{tikzpicture} \end{center}
\end{frame}

\begin{frame}{Спасибо за внимание!}
	\tableofcontents
\end{frame}

\end{document}
