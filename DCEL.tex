\section{DCEL (de Berg et al., p. 29+10)}

\begin{frame}{Карта на плоскости}
Разбиение \(\br^2\) на маркированные области.
\begin{center}
\begin{tikzpicture}[scale=0.56,draw=dcelcolor]
	\placeDcel
	\bsobl{\placeDcelEdges}
\end{tikzpicture}
\end{center}
\end{frame}

\begin{frame}{Карта на плоскости}
В вершинах и в областях могут храниться разные данные.
\begin{center}
\begin{tikzpicture}[scale=0.56,draw=dcelcolor]
	\placeDcel
	\bsobl{
	  \fill[p9260u,opacity=0.42] (v5)--(v8)--(v12)--(v14)--(v15)--(v9)--cycle;
	  \fill[p9121u,opacity=0.42] (v10)--(v16)--(v15)--(v9)--cycle;
	  \fill[p9480u,opacity=0.42] (v10)--(v9)--(v6)--cycle;
	  \fill[p9584u,opacity=0.42] (v9)--(v6)--(v3)--(v5)--cycle;
	  \fill[p9480u,opacity=0.42] (v3)--(v5)--(v2)--(v1)--cycle;
	  \fill[p9121u,opacity=0.42] (v2)--(v4)--(v8)--(v5)--cycle;
	  \fill[p9480u,opacity=0.42] (v4)--(v8)--(v7)--cycle;
	  \fill[p9584u,opacity=0.42] (v7)--(v8)--(v12)--(v11)--cycle;
	  \fill[p9121u,opacity=0.42] (v12)--(v11)--(v13)--(v14)--cycle;
	  \placeDcelEdges
	}
\end{tikzpicture}
\end{center}
\end{frame}

\begin{frame}{Поддерживаемые операции}
\begin{itemize}
	\item Обход области против часовой стрелки: даны клетка и ребро,\\
	      перейти к следующему ребру. \medskip
	\item Обход вершины по часовой стрелке: даны вершина и ребро,\\
	      перейти к следующему ребру. \medskip
	\item Переход между клетками: даны клетка и ребро, указать\\
	      клетку «по ту сторону» ребра.
\end{itemize}
\end{frame}

\begin{frame}{Полурёбра}
Рассмотрим вместо каждого ребра его два направленных варианта.
\begin{center}
\begin{tikzpicture}[scale=0.56,draw=dcelcolor]
	\placeDcel
	\placeDcelArrows
\end{tikzpicture}
\end{center}
\end{frame}

\begin{frame}{Структура данных}
Запись для каждых вершины, ребра, области; и указатели: \begin{itemize}
	\item от каждого полуребра к предыдущему и следующему,
	\item между двумя полурёбрами одного ребра,
	\item между вершиной и смежными с ней рёбрами,
	\item между областью и её рёбрами.
\end{itemize} \begin{center} \begin{tikzpicture}[scale=0.56,draw=dcelcolor]
	\dnode{0,0}{1} \dnode{4,-0.5}{2} \dnode{6,3}{3} \dnode{0.5,4}{4}
   \bsobl{
	\draw[dgray] (v1)--(v2) \ninjanode{0.6}{s1} -- (v3) \ninjanode{0.4}{s2}
	             (v3)--(v4) \ninjanode{0.45}{m1} \ninjanode{0.625}{vn}
	             (v4)--(v1) \ninjanode{0.65}{m2};
   }
	\dconn12 \dconn23 \dconn34 \dconn41
	\draw[FireBrick,<->] (s1) to[out=90,in=150] (s2);
	\draw[FireBrick,<->] (m1) to[out=255,in=50] (2.5,1.5);
	\draw[FireBrick,<->] (vn) to[out=-150,in=-60] (vn4);
	\draw[rotate=-7.1,FireBrick,<->] (m2) .. controls ++(2,1.2) and ++(-2,1.2) .. (m2);
\end{tikzpicture} \end{center}
\end{frame}

\begin{frame}{Дополнительные операции}
Имея DCEL, легко построить DCEL для {\it двойственного графа.} \\
Скопируем записи вершин, это будут новые грани.
\begin{center} \begin{tikzpicture}[scale=0.56,draw=dcelcolor]
	\dnode{0,0}{1} \dnode{4,-0.5}{2} \dnode{6,3}{3} \dnode{0.5,4}{4}
	\dconn12 \dconn23 \dconn34 \dconn41
	\fnode{7,1}{1} \fnode{2.5,1.5}{2} \fnode{1.5,-1.5}{3}
	\draw[fill3,->] (fn1) to[bend right] node[pos=0.55,inner sep=0.2mm](ve1){ }
	   node[pos=0.78,inner sep=0.2mm](ns1){ } (fn2);
	\draw[fill3,->] (fn2) to[bend right] node[pos=0.27,inner sep=0.2mm](ns2){ }
	   node[pos=0.45,inner sep=0.2mm](ve2){ } (fn3);
	\draw[FireBrick,<->] (vn2) to[bend left] (ve1);
	\draw[FireBrick,<->] (vn2) to[bend right] (ve2);
	\draw[FireBrick,<->] (ns1) to[bend left] (ns2);
\end{tikzpicture} \end{center} \vspace{-3mm}
Чтобы соединять указателями полурёбра, будем хранить,\\
какие вершины мы уже обслужили.
\end{frame}

\begin{frame}{Разрезание и склеивание}
\begin{itemize}
	\item \(\dcsplit (f,v_1,v_2)\)~— разделить область \(f\) ребром \(v_1v_2\) на две новых.
	\item \(\dcmerge (f_1,f_2,e)\)~— стереть ребро \(e\), объединив между собой \\
	   области \(f_1\) и \(f_2\).
\end{itemize}
\begin{center} \begin{tikzpicture}[scale=0.56,draw=dcelcolor]
	\dnode{0,0}{1} \dnode{4,-0.5}{2} \dnode{6,3}{3} \dnode{0.5,4}{4}
	\dconn12 \dconn23 \dconn34 \dconn41
	\draw[fill3] (vn2)--(vn4);
\end{tikzpicture} \end{center} \vspace{-3mm}

	В худшем случае занимают время~\(O(n)\), потому что переделывать \\
	все указатели ребро~\(\leftrightarrow\) область.
\end{frame}

\begin{frame}{Ломти}
	Разделим клетки на \alert{маленькие} и \alert{большие}. Большие~— те,\\
	у которых больше \(\sqrt{n}\) вершин.

	Полурёбра маленьких клеток будем хранить как есть, полурёбра\\
	больших клеток объединим в \alert{ломти} по примерно \(\sqrt{n}\) подряд идущих. \bigskip

	{\footnotesize \textcolor{white!46!dgray}{Картинка на доске}}
\end{frame}

\begin{frame}{То же самое за polylog}
\begin{block}{Упражнение.}
	Убедитесь, что, если хранить полурёбра в структуре\\
	{\it link-cut tree,} можно делать операции split и merge\\
	в среднем за \(O\lr*{\text{polylog}\lr*{n}}\).
\end{block}
\end{frame}
