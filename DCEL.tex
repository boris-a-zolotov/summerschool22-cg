\documentclass[12pt,aspectratio=169,svgnames]{beamer}

\usepackage{modules/cgs}

\begin{document} \maketitle

\begin{frame}{Содержание}
	\tableofcontents
\end{frame}

\section{DCEL (de Berg et al., p.\,\,29\,\(+\)\,10)}

\begin{frame}{Карта на плоскости}
Разбиение \(\br^2\) на маркированные области.
\begin{center}
\begin{tikzpicture}[scale=0.56,draw=dcelcolor]
	\placeDcel
	\bsobl{\placeDcelEdges}
\end{tikzpicture}
\end{center}
\end{frame}

\begin{frame}{Карта на плоскости}
В вершинах и в областях могут храниться разные данные.
\begin{center}
\begin{tikzpicture}[scale=0.56,draw=dcelcolor]
	\placeDcel
	\begin{scope}[on background layer]
		\fill[red] (v5)--(v8)--(v12)--(v14)--(v15)--(v9)--cycle;
		\placeDcelEdges
	\end{scope}
\end{tikzpicture}
\end{center}
\end{frame}
\end{document}
